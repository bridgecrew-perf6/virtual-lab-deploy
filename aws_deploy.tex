\textbf{Implantação em Amazon Web Services (AWS) utilizado recurso EC2
de planta virtual utilizando node-red e supervisório utilizando
Scada-LTS.}

\begin{enumerate}
\def\labelenumi{\arabic{enumi}.}
\item
  Pré-requisitos

  \begin{enumerate}
  \def\labelenumii{\arabic{enumii}.}
  \item
    Ter conta no Amazon Web Services (AWS): \url{https://aws.amazon.com}
  \item
    Conhecimento básico em informática, git e execução de comandos
    shell.
  \end{enumerate}
\item
  Clonar o repositório no GitHub:
  \url{https://github.com/rlmariz/virtual-lab-deploy.git}
\item
  Após clonar o repositório a pasta de trabalho será
  \textbf{``virtual-lab-deploy\textbackslash aws''}, todos os arquivos
  salvos e comandos executados devem ser nessa pasta.
\item
  Criar conta de acesso para ser utilizado o terraform no AWS

  \begin{enumerate}
  \def\labelenumii{\arabic{enumii}.}
  \tightlist
  \item
    Acesse o console da AWS e faça o login com sua conta e pesquise pelo
    produto IAM (Identity and Access Management).
  \end{enumerate}

  \begin{enumerate}
  \def\labelenumii{\arabic{enumii}.}
  \setcounter{enumii}{1}
  \tightlist
  \item
    Iremos criar um usuário para que o terraform possa interagir com a
    AWS, clique em USERS e em seguida em ADD USER.
  \end{enumerate}

  \begin{enumerate}
  \def\labelenumii{\arabic{enumii}.}
  \setcounter{enumii}{2}
  \tightlist
  \item
    Definir detalhes do usuário.
  \end{enumerate}

  \begin{enumerate}
  \def\labelenumii{\arabic{enumii}.}
  \setcounter{enumii}{3}
  \tightlist
  \item
    Adicione a política AmazonEC2FullAccess ao usuário, o que dará
    permissão total ao usuário apenas a recursos da EC2, e clique em
    Next.
  \end{enumerate}

  \begin{enumerate}
  \def\labelenumii{\arabic{enumii}.}
  \setcounter{enumii}{4}
  \tightlist
  \item
    Tags são utilizadas para adicionar informações relevantes ao
    usuario, clique em Next.
  \end{enumerate}

  \begin{enumerate}
  \def\labelenumii{\arabic{enumii}.}
  \setcounter{enumii}{5}
  \tightlist
  \item
    Verifique os dados e clique em Create user.
  \end{enumerate}

  \begin{enumerate}
  \def\labelenumii{\arabic{enumii}.}
  \setcounter{enumii}{6}
  \tightlist
  \item
    Clique em show e copie o Access key ID e Secret access key
  \end{enumerate}

  \begin{enumerate}
  \def\labelenumii{\arabic{enumii}.}
  \setcounter{enumii}{7}
  \tightlist
  \item
    O usuário criado e chave de acesso não devem ser compartilhados, uma
    vez que quem tiver acesso a estes dados terá controle sobre os
    recursos adicionados como política.
  \end{enumerate}
\item
  Editar arquivo \textbf{\emph{aws\_credentials.txt}} e adicionar a
  chave de acesso e a chave secreta substituindo os valores
  \textbf{\textless access\_key\textgreater{}} e
  \textbf{\textless secret\_access\_key\textgreater{}}.
\item
  Acessar o site e gerar par de chaves rsa que será utilizado para
  conexão ssh.

  \begin{enumerate}
  \def\labelenumii{\arabic{enumii}.}
  \item
    Acesso o website
    \href{https://www.wpoven.com/tools/create-ssh-key}{https://www.wpoven.com/tools/create-ssh-key\#}
  \item
    Configure o type como rsa, length 2048, password deixe em branco e
    clique em create key.
  \end{enumerate}

  \begin{enumerate}
  \def\labelenumii{\arabic{enumii}.}
  \setcounter{enumii}{2}
  \item
    Fazer download do \textbf{\emph{Private Key}} e salvar arquivo com
    nome \textbf{\emph{aws.key}}.
  \item
    Fazer download do \textbf{\emph{Public Key}} e salvar arquivo com
    nome \textbf{\emph{aws.pub.key}}.
  \end{enumerate}
\item
  Instalar o Terraform

  \begin{enumerate}
  \def\labelenumii{\arabic{enumii}.}
  \item
    Acesse o site \url{https://www.terraform.io/downloads}.
  \item
    Siga as instruções de acordo o sistema operacional que está
    utilizando.
  \item
    Ao fazer o download do executável de preferência coloque na pasta de
    trabalho para facilitar sua utilização.
  \item
    Pode ser feito o teste para verificar se está tudo ok executando no
    prompt de comandos:
  \item
    Vai exibir a versão instalada e a plataforma:
  \end{enumerate}
\item
  Caso seja necessário pode se editar o arquivo variables.tf e fazer os
  ajustes necessários.
\end{enumerate}

\begin{enumerate}
\def\labelenumi{\arabic{enumi}.}
\setcounter{enumi}{8}
\tightlist
\item
  Executar o comando para instalar os requisitos do terraform.
\end{enumerate}

\begin{enumerate}
\def\labelenumi{\arabic{enumi}.}
\setcounter{enumi}{9}
\tightlist
\item
  O próximo passo é criar infraestrutura e subir aplicação, vamos
  executar o comando:
\end{enumerate}

\begin{enumerate}
\def\labelenumi{\arabic{enumi}.}
\setcounter{enumi}{10}
\tightlist
\item
  Vai ser exibido o plano de trabalho e estando tudo ok basta digitar
  ``yes'' e dar enter.
\end{enumerate}

\begin{enumerate}
\def\labelenumi{\arabic{enumi}.}
\setcounter{enumi}{11}
\item
  O processo leva certa de 6 minutos para ser implementado e ao final
  será exibido os endpoints para acesso ao supervisório, node-red e caso
  necessário o comando para acesso via ssh.
\item
  O usuário e senha para acesso ao supervisório é
  \textbf{\emph{admin/admin}}.
\item
  É muito importante desalocar os recursos após finalizar sua utilização
  para não ter custos extras, para fazer isso basta executar o comando
  abaixo e confirmar com yes, ao final os recursos serão desalocado.
\end{enumerate}
